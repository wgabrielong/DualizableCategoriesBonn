\part*{A. Matthew -- Dualizable Categories and their K-Theory}
\section{Lecture I (9th September)}\label{sec: Matthew I}
We refer to Lurie's foundational texts \cite{HTT} and \cite{HA} for background on $\infty$-categories and higher algebra. 

Recall that the $\infty$-categorical analogue of an Abelian category is a stable $\infty$-category. Grothendieck groups can be defined for stable $\infty$-categories analogously to how Grothendieck groups are defined for Abelian categories. 
\begin{definition}[Grothendieck Group]\label{def: Grothendieck group}
    Let $\Cscr$ be a stable $\infty$-category. The Grothendieck group $\ksf_{0}(\Cscr)$ of $\Cscr$ is defined to be the quotient
    $$\ksf_{0}(\Cscr)=\frac{\ZZ\left[\left\{[X]:X\in\Cscr\right\}\right]}{\sim}$$
    where $[X]\sim[X']+[X'']$ if there exists a cofiber sequence $X'\to X\to X''$ in $\Cscr$. 
\end{definition}
\begin{remark}
    The construction of the Grothendieck group depends only on the underlying homotopy category $\hsf\Cscr$. The homotopy category $\hsf\Cscr$ is in fact triangulated by \cite[Thm. 1.1.2.4]{HA}.
\end{remark}
A more contemporary interpretation of algebraic K-theory following work of Quillen, Waldhausen, and others, is to define K-theory as a spectrum $K(\Cscr)$ such that $\pi_{0}\left(K(\Cscr)\right)=\ksf_{0}(\Cscr)$. In some sense, this is the ``right  homotopical enhancement'' of $\ksf_{0}(-)$. Formalizing how this is the ``right homotopical enhancement'' will require a discussion of localizing invariants. 

Recall the following definitions. 
\begin{definition}[Idempotent Complete]\label{def: idempotent complete}
    Let $\Cscr$ be an $\infty$-category. $\Cscr$ is idempotent complete if its image under the Yoneda embedding $\Cscr\to\Fun(\Cscr^{\Opp},\Grpd_{\infty})$ is closed under retracts. 
\end{definition}
\begin{definition}[Exact Functor]\label{def: exact functor}
    Let $f:\Escr\to\Cscr$ be a functor between $\infty$-categories. $f$ is an exact functor if $f$ preserves finite colimits. 
\end{definition}
As such, we make the following definitions. 
\begin{definition}[$\Cat^{\Perf}$]\label{def: CatPerf}
    $\Cat^{\Perf}$ is the $\infty$-category of small idempotent-complete $\infty$-categories and exact functors. 
\end{definition}
\begin{remark}
    Note that exact functors preserve finite colimits. 
\end{remark}
To define localizing invariants, we need to discuss Karoubi quotients, which are the analogue of Verdier quotients in $\Cat^{\Perf}$. 
\begin{definition}[Karoubi Quotient]\label{def: Karoubi quotient}
    Let $\Cscr\in\Cat^{\Perf}$ and $\Bscr\subseteq\Cscr$ an inclusion in $\Cat^{\Perf}$. The Karoubi quotient $\Cscr/\Bscr$ is defined to be the pushout in $\Cat^{\Perf}$
    \begin{equation}\label{diag: Karoubi quotient}
        % https://q.uiver.app/#q=WzAsNCxbMCwwLCJcXEJzY3IiXSxbMiwwLCJcXENzY3IiXSxbMCwxLCIwIl0sWzIsMSwiMFxcYmlnY3VwX3tcXEJzY3J9XFxDc2NyPVxcQ3Njci9cXEJzY3IiXSxbMCwxXSxbMSwzXSxbMCwyXSxbMiwzXV0=
    \begin{tikzcd}
        \Bscr && \Cscr \\
        0 && {0\bigcup_{\Bscr}\Cscr=\Cscr/\Bscr.}
        \arrow[from=1-1, to=1-3]
        \arrow[from=1-1, to=2-1]
        \arrow[from=1-3, to=2-3]
        \arrow[from=2-1, to=2-3]
    \end{tikzcd}
    \end{equation}
\end{definition}
\begin{remark}
    The pushout square (\ref{diag: Karoubi quotient}) is also a pullback square. 
\end{remark}
Informally, we can think of the Karoubi quotient using the quotient functor $p:\Cscr\to\Cscr/\Bscr$ where given $x,y\in\Cscr$ the hom-objects in the quotient can be computed by 
$$\Hom_{\Cscr/\Bscr}(p(x),p(y))=\colim\Hom_{\Cscr}(x,y')$$
where $y\to y'$ is a morphism in $\Cscr$ such that its cofiber $\cofib(y\to y')\in\Bscr$. 
\begin{remark}
    The quotient functor $p:\Cscr\to\Cscr/\Bscr$ is not essentially surjective but it is up to retracts/idempotent completion. As a consequence of a result of Thomason, a class $Z\in\Cscr/\Bscr$ lifts if and only if the corresponding class in the Grothendieck group lifts. As such, $Z\oplus Z[1]$ always lifts. 
\end{remark}
\begin{example}
    Let $X$ be a quasicompact and quasiseparated scheme and $U\subseteq X$ a quasicompact open. By arguments of Thomason-Trabough there is a Karoubi sequence 
    $$\Perf(X\setminus U)\longrightarrow\Perf(X)\longrightarrow\Perf(U).$$
\end{example}
Another important example of Karoubi quotients is that of the Calkin category which we define as follows. 
\begin{definition}[Calkin Category]\label{def: Calkin category}
    Let $\Ascr\in\Cat^{\Perf}$ and $\kappa$ a regular cardinal. The Calkin category $\Calk_{\kappa}(\Ascr)$ is defined to be the Karoubi quotient $\Ind(\Ascr)^{\kappa}/\Ascr$ of the $\kappa$-compact objects of $\Ind(\Ascr)$ by $\Ascr$. 
\end{definition}
\begin{remark}
    The most relevant case will be when $\kappa=\omega_{1}$ in which case the Calkin category will be the quotient of the subcategory of $\Ind(\Ascr)$ generated by sequential colimits modulo the constant colimits. 
\end{remark}
Denoting $\lim^{\kappa},\colim^{\kappa}$ $\kappa$-filtered (co)limits, we can compute hom-objects in the Calkin category as the quotient 
$$\Hom_{\Calk(\Ascr)}(\colim^{\kappa}_{i\in I}x_{i}, \colim^{\kappa}_{j\in J}y_{j})=\frac{\lim_{i\in I}\colim_{j\in J}\Hom_{\Ascr}(x_{i},y_{j})}{\colim_{j\in J}\lim_{i\in I}\Hom_{\Ascr}(x_{i},y_{j})}.$$
\begin{example}
    Let $R$ be a ring. For $\Ascr=\Perf(R)$, $\Calk(\Ascr)=\Dscr(R)/\Perf(R)$ and 
    $$\Hom_{\Calk(\Ascr)}(M,M')=\frac{\Hom_{\Mod_{R}}(M,M')}{\Hom_{\Mod_{R}}(M,R)\otimes_{R}M'}.$$
    If $R=k$ a field and $M,M'$ $k$-vector space then the morphisms in the Calkin category would be the quotient of all linear maps $M\to M'$ by the linear maps of finite rank. 
\end{example}
Let us turn to semiorthogonal decompositions of $\infty$-categories. 
\begin{definition}[Semiorthogonal Decomposition]\label{def: semoirthogonal decomposition}
    Let $\Cscr$ be a stable idempotent complete $\infty$-category. A semiorthogonal decomposition of $\langle\Cscr_{1},\Cscr_{2}\rangle$ of $\Cscr$ is the data of stable subcategories $\Cscr_{1},\Cscr_{2}$ such that 
    \begin{enumerate}[label=(\roman*)]
        \item If $x_{1}\in\Cscr_{1}, x_{2}\in\Cscr_{2}$ then $\Hom_{\Cscr}(x_{2},x_{1})=0$. 
        \item For all $y\in\Cscr$ there is a fiber sequence $y_{2}\to y\to y_{1}$ with $y_{1}\in\Cscr_{1},y_{2}\in\Cscr_{2}$. 
    \end{enumerate}
\end{definition}
\begin{remark}
    For (ii) of \Cref{def: semoirthogonal decomposition}, $y$ uniquely determines contractible choices of $y_{1},y_{2}$ by (i). It is enough to require that $\Cscr_{1},\Cscr_{2}$ generate $\Cscr$ as a stable $\infty$-category and this stable generation property is in fact equivalent to (ii). 
\end{remark} 
Given a stable idempotent complete $\infty$-category $\Cscr$ with semiorthogonal decomposition $\Cscr=\langle\Cscr_{1},\Cscr_{2}\rangle$ there is a Karoubi sequence 
$$% https://q.uiver.app/#q=WzAsMyxbMCwwLCJcXENzY3JfezJ9Il0sWzIsMCwiXFxDc2NyIl0sWzQsMCwiXFxDc2NyX3sxfSJdLFswLDEsIiIsMCx7InN0eWxlIjp7InRhaWwiOnsibmFtZSI6Imhvb2siLCJzaWRlIjoidG9wIn19fV0sWzEsMl1d
\begin{tikzcd}
	{\Cscr_{2}} && \Cscr && {\Cscr_{1}}
	\arrow[hook, from=1-1, to=1-3]
	\arrow[from=1-3, to=1-5]
\end{tikzcd}$$
where the map $\Cscr\to\Cscr_{1}$ is given by $y\mapsto y_{1}$ as in \Cref{def: semoirthogonal decomposition} (ii). 

More generally, given a Karoubi sequence 
\begin{equation}\label{diag: Karoubi seq}
    % https://q.uiver.app/#q=WzAsMyxbMCwwLCJcXEJzY3IiXSxbMiwwLCJcXENzY3IiXSxbNCwwLCJcXENzY3IvXFxCc2NyIl0sWzAsMSwiaSJdLFsxLDIsInAiXV0=
\begin{tikzcd}
	\Bscr && \Cscr && {\Cscr/\Bscr}
	\arrow["i", from=1-1, to=1-3]
	\arrow["p", from=1-3, to=1-5]
\end{tikzcd}
\end{equation}
and either $i$ admits a right adjoint $i^{R}$ or $p$ admits a right adjoint $p^{R}$ then $\Cscr=\langle\ker(i^{R}), i(\Bscr)\rangle$ or $\Cscr=\langle p^{R}(\Cscr/\Bscr), i(\Bscr)\rangle$, respectively. 
\begin{example}
    $\Perf(\mathbb{P}^{1}_{k})=\langle\Ocal_{\mathbb{P}^{1}_{k}},\Ocal_{\mathbb{P}^{1}_{k}}(1)\rangle$. 
\end{example}
Revisiting Karoubi quotients, we would expect Karoubi quotients to remain so after ind-completion. Under ind-completion functors have right adjoints by the adjoint functor theorem and as such we would expect that for a Karoubi sequence as in (\ref{diag: Karoubi seq}) above, the category $\Cscr$ will admit a semiorthogonal decomposition described using these adjoint functors. In particular, ind-completion induces a Karoubi sequence
$$% https://q.uiver.app/#q=WzAsMyxbMCwwLCJcXEluZChcXEJzY3IpIl0sWzIsMCwiXFxJbmQoXFxDc2NyKSJdLFs0LDAsIlxcSW5kKFxcQ3Njci9cXEJzY3IpIl0sWzAsMSwiaSJdLFsxLDIsInAiXV0=
\begin{tikzcd}
	{\Ind(\Bscr)} && {\Ind(\Cscr)} && {\Ind(\Cscr/\Bscr)}
	\arrow["i", from=1-1, to=1-3]
	\arrow["p", from=1-3, to=1-5]
\end{tikzcd}$$
from which we conclude that $\Cscr/\Bscr=\ker(i^{R}:\Ind(\Cscr)\to\Dscr)^{\omega}$, the compact objects of the kernel of the right adjoint of $i$. 
\begin{remark}
    There is an equivalence of $\infty$-categories $\Ind(\Cscr)\to\Fun^{\Ex}(\Cscr^{\Opp}, \Sp)$ by $\colim^{\kappa}_{i\in I}x_{i}\mapsto\colim_{i\in I}\ysf_{x_{i}}$ with $\ysf_{i}$ the spectra-valued Yoneda functor $\Hom_{\Cscr}(-,x_{i})$ and the right adjoint given by restriction. This construction is analogous to the computation of colimits in $\catPr^{\Lsf}$ as limits on passage to right adjoints. 
\end{remark} 
We can now define localizing invariants. 
\begin{definition}[Localizing Invariant]\label{def: localizing invariant}
    Let $\Escr$ be a stable $\infty$-category. A localizing invariant is a functor $F:\Cat^{\Perf}\to\Escr$ such that:
    \begin{enumerate}[label=(\roman*)]
        \item $F(0)=0$.
        \item If $\Bscr\subseteq\Cscr$ is an inclusion in $\Cat^{\Perf}$ then 
        $$% https://q.uiver.app/#q=WzAsNCxbMCwwLCJGKFxcQnNjcikiXSxbMiwwLCJGKFxcQ3NjcikiXSxbMiwxLCJGKFxcQ3Njci9cXEJzY3IpIl0sWzAsMSwiMCJdLFswLDNdLFszLDJdLFsxLDJdLFswLDFdXQ==
        \begin{tikzcd}
            {F(\Bscr)} && {F(\Cscr)} \\
            0 && {F(\Cscr/\Bscr)}
            \arrow[from=1-1, to=1-3]
            \arrow[from=1-1, to=2-1]
            \arrow[from=1-3, to=2-3]
            \arrow[from=2-1, to=2-3]
        \end{tikzcd}$$
        is a pushout. 
    \end{enumerate}
\end{definition}
\begin{remark}
    One usually requires $\Escr$ to be accessible and $F$ to preserve $\kappa$-filtered colimits for some regular cardinal $\kappa$. 
\end{remark}
Localizing invariants give rise to additive invariants via semiorthogonal decompositions. 
\begin{example}
   Let $\Cscr\in\Cat^{\Perf}$ with semiorthogonal decomposition $\Cscr=\langle\Cscr_{1},\Cscr_{2}\rangle$. If $F$ is a localizing invariant then there is an isomorphism $F(\Cscr_{1})\oplus F(\Cscr_{2})\to F(\Cscr)$. Functors satisfying this condition are known as additive invariants. 
\end{example}
\begin{example}
    Let $\Cscr\in\Cat^{\Perf}$ and $F$ a localizing invariant. Consider $\Ind(\Cscr)^{\omega_{1}}$ and $\Calk_{\omega_{1}}(\Cscr)=\Ind(\Cscr)^{\omega_{1}}/\Cscr$. There is a fiber sequence 
    $$% https://q.uiver.app/#q=WzAsMyxbMCwwLCJGKFxcQ3NjcikiXSxbMiwwLCJGKFxcSW5kKFxcQ3Njcilee1xcb21lZ2FfezF9fSkiXSxbNCwwLCJGKFxcQ2Fsa197XFxvbWVnYV97MX19KFxcQ3NjcikpIl0sWzAsMV0sWzEsMl1d
    \begin{tikzcd}
        {F(\Cscr)} && {F(\Ind(\Cscr)^{\omega_{1}})} && {F(\Calk_{\omega_{1}}(\Cscr)).}
        \arrow[from=1-1, to=1-3]
        \arrow[from=1-3, to=1-5]
    \end{tikzcd}$$
    However, by the Eilenberg Swindle, the middle term vanishes so $F(\Cscr)$ is $\Omega F(\Calk_{\omega_{1}}(\Cscr))$.
\end{example}
Returning to K-theory, K-theory can in fact be defined as a universal additive invariant. 
\begin{theorem}[Blumberg-Gepner-Tabuada, Barwick]\label{thm: initial localizing invariant}
    The K-theory functor $K:\Cat^{\Perf}\to\Sp$ is the initial localizing invariant with a natural map from the underlying groupoid $\Cscr^{\simeq}\to\Omega^{\infty}K(\Cscr)$ for all $\Cscr\in\Cat^{\Perf}$. 
\end{theorem}
Other well-known examples of localizing invariants are topological Hochschild homology and topological cyclic homology. 
\begin{definition}[Localizing Motives]\label{def: localizing motives}
    $\MotLoc$ is the initial stable presentable $\infty$-category equipped with a localizing invariant $\Ucal_{\Loc}:\Cat^{\Perf}\to\MotLoc$ that preserves filtered colimits. 
\end{definition}
In other words, $\MotLoc$ can be obtained by a quotient of the ``free stable presentable $\infty$-category'' $\Fun((\Cat^{\Perf, \omega})^{\Opp},\Sp)$. 
\begin{theorem}[Blumberg-Gepner-Tabuada]\label{thm: localizingm motives is k theory}
    For all $\Cscr\in\Cat^{\Perf}$, 
    $$\Hom_{\MotLoc}(\Ucal_{\Loc}(\Sp^{\omega}), \Ucal_{\Loc}(\Cscr))=K(\Cscr).$$
\end{theorem}