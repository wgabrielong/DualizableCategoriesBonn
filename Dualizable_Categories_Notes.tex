\documentclass{amsart}
\usepackage[margin=1.1in]{geometry} 
\usepackage{amsmath}
\usepackage{tcolorbox}
\usepackage{amssymb}
\usepackage{amsthm}
\usepackage{lastpage}
\usepackage{fancyhdr}
\usepackage{accents}
\usepackage{hyperref}
\usepackage{xcolor}
\usepackage{color}
\input{shortcuts.tex}
\setlength{\headheight}{40pt}


\newenvironment{solution}
  {\renewcommand\qedsymbol{$\blacksquare$}
  \begin{proof}[Solution]}
  {\end{proof}}
\renewcommand\qedsymbol{$\blacksquare$}

\usepackage{amsmath, amssymb, tikz, amsthm, csquotes, multicol, footnote, tablefootnote, biblatex, wrapfig, float, quiver, mathrsfs, cleveref, enumitem, upgreek}
\addbibresource{refs.bib}
\theoremstyle{definition}
\newtheorem{theorem}{Theorem}[section]
\newtheorem{lemma}[theorem]{Lemma}
\newtheorem{corollary}[theorem]{Corollary}
\newtheorem{exercise}[theorem]{Exercise}
\newtheorem{question}[theorem]{Question}
\newtheorem{example}[theorem]{Example}
\newtheorem{proposition}[theorem]{Proposition}
\newtheorem{conjecture}[theorem]{Conjecture}
\newtheorem{remark}[theorem]{Remark}
\newtheorem{definition}[theorem]{Definition}
\numberwithin{equation}{section}
\setcounter{tocdepth}{1}
\begin{document}
\large
\title[Dualizable Categories and Continuous K-Theory -- MPIM 2024]{Workshop on Dualizable Categories and Continuous K-Theory \\ Max Planck Institute for Mathematics, Bonn -- July 2024}
\author{Wern Juin Gabriel Ong}
\address{Universit\"{a}t Bonn, Bonn, D-53111}
\email{wgabrielong@uni-bonn.de}
\urladdr{https://wgabrielong.github.io/}
\maketitle
\section*{Preliminaries}
This document contains notes from the 2024 Workshop on Dualizable Categories and Continuous K-Theory held at the Max Planck Institute for Mathematics in Bonn in September 2024. Short courses were given by A. Matthew (Chicago) and A. Efimov (HUJI). 
This document contains notes from the 2024 Workshop on Dualizable Categories and Continuous K-Theory held at the Max Planck Institute for Mathematics in Bonn in September 2024. Short courses were given by A. Matthew (Chicago) and A. Efimov (HUJI). These notes are \LaTeX-ed after the fact with significant alteration and are subject to misinterpretation and mistranscription. Use with caution. Any errors are undoubtedly my own and any virtues ought to be attributed to the speakers and not the typist. 
\tableofcontents
\newpage
\part*{End Matter}
\printbibliography
\end{document}
