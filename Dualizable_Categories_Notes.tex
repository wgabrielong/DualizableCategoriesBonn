\documentclass{amsart}
\usepackage[margin=1.1in]{geometry} 
\usepackage{amsmath}
\usepackage{tcolorbox}
\usepackage{amssymb}
\usepackage{amsthm}
\usepackage{lastpage}
\usepackage{fancyhdr}
\usepackage{accents}
\usepackage{hyperref}
\usepackage{xcolor}
\usepackage{color}
% Fields
\newcommand{\CC}{\mathbb{C}}
\newcommand{\RR}{\mathbb{R}}
\newcommand{\QQ}{\mathbb{Q}}
\newcommand{\ZZ}{\mathbb{Z}}
\newcommand{\NN}{\mathbb{N}}

% mathcal letters
\newcommand{\Acal}{\mathcal{A}}
\newcommand{\Bcal}{\mathcal{B}}
\newcommand{\Ccal}{\mathcal{C}}
\newcommand{\Dcal}{\mathcal{D}}
\newcommand{\Ecal}{\mathcal{E}}
\newcommand{\Fcal}{\mathcal{F}}
\newcommand{\Gcal}{\mathcal{G}}
\newcommand{\Hcal}{\mathcal{H}}
\newcommand{\Ical}{\mathcal{I}}
\newcommand{\Jcal}{\mathcal{J}}
\newcommand{\Kcal}{\mathcal{K}}
\newcommand{\Lcal}{\mathcal{L}}
\newcommand{\Mcal}{\mathcal{M}}
\newcommand{\Ncal}{\mathcal{N}}
\newcommand{\Ocal}{\mathcal{O}}
\newcommand{\Pcal}{\mathcal{P}}
\newcommand{\Qcal}{\mathcal{Q}}
\newcommand{\Rcal}{\mathcal{R}}
\newcommand{\Scal}{\mathcal{S}}
\newcommand{\Tcal}{\mathcal{T}}
\newcommand{\Ucal}{\mathcal{U}}
\newcommand{\Vcal}{\mathcal{V}}
\newcommand{\Wcal}{\mathcal{W}}
\newcommand{\Xcal}{\mathcal{X}}
\newcommand{\Ycal}{\mathcal{Y}}
\newcommand{\Zcal}{\mathcal{Z}}

% abstract categories
\newcommand{\Asf}{\mathsf{A}}
\newcommand{\Bsf}{\mathsf{B}}
\newcommand{\Csf}{\mathsf{C}}
\newcommand{\Dsf}{\mathsf{D}}
\newcommand{\hsf}{\mathsf{h}}
\newcommand{\Rsf}{\mathsf{R}}
\newcommand{\Ssf}{\mathsf{S}}
\newcommand{\Tsf}{\mathsf{T}}
\newcommand{\Lsf}{\mathsf{L}}
\newcommand{\ksf}{\mathsf{k}}
\newcommand{\ysf}{\mathsf{y}}

% infinity categories
\newcommand{\Ascr}{\EuScript{A}}
\newcommand{\Bscr}{\EuScript{B}}
\newcommand{\Cscr}{\EuScript{C}}
\newcommand{\Dscr}{\EuScript{D}}
\newcommand{\Escr}{\EuScript{E}}
\newcommand{\catPr}{\mathsf{Pr}}

% algebraic geometry
\newcommand{\spec}{\operatorname{Spec}}
\newcommand{\proj}{\operatorname{Proj}}

% categories 
\newcommand{\id}{\mathrm{id}}
\newcommand{\Obj}{\mathrm{Obj}}
\newcommand{\Mor}{\mathrm{Mor}}
\newcommand{\Hom}{\mathrm{Hom}}
\newcommand{\Aut}{\mathrm{Aut}}
\newcommand{\Sets}{\mathsf{Sets}}
\newcommand{\SSets}{\mathsf{SSets}}
\newcommand{\kVec}{\mathsf{Vec}_{k}}
\newcommand{\Alg}{\mathsf{Alg}}
\newcommand{\Ring}{\mathsf{Ring}}
\newcommand{\Mod}{\mathsf{Mod}}
\newcommand{\Grp}{\mathsf{Grp}}
\newcommand{\AbGrp}{\mathsf{AbGrp}}
\newcommand{\PSh}{\mathsf{PSh}}
\newcommand{\Sh}{\mathsf{Sh}}
\newcommand{\PSch}{\mathsf{PSch}}
\newcommand{\Sch}{\mathsf{Sch}}
\newcommand{\Top}{\mathsf{Top}}
\newcommand{\Com}{\mathsf{Com}}
\newcommand{\Coh}{\mathsf{Coh}}
\newcommand{\QCoh}{\mathsf{QCoh}}
\newcommand{\Opens}{\mathsf{Opens}}
\newcommand{\Opp}{\mathsf{Opp}}
\newcommand{\Cat}{\mathsf{Cat}}
\newcommand{\colim}{\operatorname{colim}}
\newcommand{\Grpd}{\mathsf{Grpd}}
\newcommand{\Fun}{\mathrm{Fun}}
\newcommand{\cofib}{\mathrm{cofib}}


% simplicial sets
\newcommand{\DDelta}{\Updelta}
\newcommand{\Sing}{\operatorname{Sing}}

% condensed math
\newcommand{\LCA}{\mathsf{LCA}}
\newcommand{\Cond}{\mathsf{Cond}}
\newcommand{\dom}{\operatorname{dom}}
\newcommand{\Cov}{\operatorname{Cov}}
\newcommand{\proet}{\mathsf{pro}\mathsf{\acute{e}}\mathsf{t}}
\newcommand{\et}{\mathsf{\acute{e}t}}
\newcommand{\Ab}{\mathsf{Ab}}
\newcommand{\ProFin}{\mathsf{ProFin}}
\newcommand{\ExtDiscHaus}{\mathsf{ExtDiscHaus}}
\newcommand{\CHaus}{\mathsf{CHaus}}

% K-theory
\newcommand{\Perf}{\mathsf{Perf}}
\newcommand{\MotLoc}{\mathsf{MotLoc}}
\newcommand{\St}{\mathsf{St}}
\newcommand{\Ex}{\mathsf{Ex}}
\newcommand{\Calk}{\mathsf{Calk}}
\newcommand{\Ind}{\mathsf{Ind}}
\newcommand{\Sp}{\mathsf{Sp}}
\newcommand{\Loc}{\mathsf{Loc}}
\newcommand{\ev}{\mathrm{ev}}
\newcommand{\coev}{\mathrm{coev}}
\newcommand{\dual}{\mathsf{dual}}
\setlength{\headheight}{40pt}


\newenvironment{solution}
  {\renewcommand\qedsymbol{$\blacksquare$}
  \begin{proof}[Solution]}
  {\end{proof}}
\renewcommand\qedsymbol{$\blacksquare$}

\usepackage{amsmath, amssymb, tikz, amsthm, csquotes, multicol, footnote, tablefootnote, biblatex, wrapfig, float, quiver, mathrsfs, cleveref, enumitem, upgreek}
\addbibresource{refs.bib}
\theoremstyle{definition}
\newtheorem{theorem}{Theorem}[section]
\newtheorem{lemma}[theorem]{Lemma}
\newtheorem{corollary}[theorem]{Corollary}
\newtheorem{exercise}[theorem]{Exercise}
\newtheorem{question}[theorem]{Question}
\newtheorem{example}[theorem]{Example}
\newtheorem{proposition}[theorem]{Proposition}
\newtheorem{conjecture}[theorem]{Conjecture}
\newtheorem{remark}[theorem]{Remark}
\newtheorem{definition}[theorem]{Definition}
\numberwithin{equation}{section}
\setcounter{tocdepth}{1}
\begin{document}
\large
\title[Dualizable Categories and Continuous K-Theory -- MPIM 2024]{Workshop on Dualizable Categories and Continuous K-Theory \\ Max Planck Institute for Mathematics, Bonn -- September 2024}
\author{Wern Juin Gabriel Ong}
\address{Universit\"{a}t Bonn, Bonn, D-53111}
\email{wgabrielong@uni-bonn.de}
\urladdr{https://wgabrielong.github.io/}
\maketitle
\section*{Preliminaries}
This document contains notes from the 2024 Workshop on Dualizable Categories and Continuous K-Theory held at the Max Planck Institute for Mathematics in Bonn in September 2024. Short courses were given by A. Efimov (HUJI) and A. Matthew (Chicago). These notes are \LaTeX-ed after the fact with significant alteration and are subject to misinterpretation and mistranscription. Use with caution. Any errors are undoubtedly my own and any virtues ought to be attributed to the speakers and not the typist. 
\tableofcontents
\newpage
\part*{A. Efimov -- Dualizable Categories and Localizing Motives}
\section{Lecture I (10th September)}\label{sec: Efimov I}
We show that dualizability is equivalent to flatness. We work here over the absolute base but the same proof applies over any rigid category. 

Recall the following definition. 
\begin{definition}[Flat]\label{def: flat}
    Let $\Cscr\in\catPr^{\Lsf}_{\St}$. $\Cscr$ is flat if for all fully faithful continuous functors $\Bscr\to\Escr$, $\Bscr\otimes\Cscr\to\Escr\otimes\Bscr$ is fully faithful. 
\end{definition}
We introduce some language necessary to show the desired result. 
\begin{definition}[$\catPr^{\acc}_{\St}$]\label{def: PrLacc}
    $\catPr^{\acc}_{\St}$ is the $\infty$-categyr with objects presentable stable $\infty$-categories and exact accessible functors. 
\end{definition}
\begin{remark}
    There is a non-full inclusion $\catPr^{\Lsf}_{\St}\hookrightarrow\catPr^{\Lsf}_{\acc}$. 
\end{remark}

\begin{definition}[Oplax 2-Functor]\label{def: oplax 2-functor}
    Let $F:\Bscr\to\Escr$ be an accessible exact functor in $\catPr^{\acc}_{\St}$ and $\Cscr\in\catPr^{\Lsf}_{\St}$. An oplax 2-functor $\Cscr\otimes\Bscr\to\Cscr\otimes\Escr$ is the given by the composition 
    $$% https://q.uiver.app/#q=WzAsNCxbMCwwLCJcXENzY3JcXG90aW1lc1xcRHNjcj1cXEZ1bl57XFxrYXBwYS1cXGxleH0oKFxcQ3Njcl57XFxrYXBwYX0pXntcXE9wcH0sXFxCc2NyKSJdLFsxLDAsIlxcRnVuKChcXENzY3Jee1xca2FwcGF9KV57XFxPcHB9LFxcQnNjcikiXSxbMiwwLCJcXEZ1bigoXFxDc2NyXntcXGthcHBhfSlee1xcT3BwfSxcXEVzY3IpIl0sWzMsMCwiXFxGdW5ee1xca2FwcGEtXFxsZXh9KChcXENzY3Jee1xca2FwcGF9KV57XFxPcHB9LFxcRXhjcik9XFxDc2NyXFxvdGltZXNcXEVzY3IiXSxbMCwxLCIiLDAseyJzdHlsZSI6eyJ0YWlsIjp7Im5hbWUiOiJob29rIiwic2lkZSI6InRvcCJ9fX1dLFsxLDIsIkZcXGNpcmMtIl0sWzIsMywiaV57TH0iXV0=
    \begin{tikzcd}
        {\Fun^{\kappa-\lex}((\Cscr^{\kappa})^{\Opp},\Bscr)} & {\Fun((\Cscr^{\kappa})^{\Opp},\Bscr)} & {\Fun((\Cscr^{\kappa})^{\Opp},\Escr)} & {\Fun^{\kappa-\lex}((\Cscr^{\kappa})^{\Opp},\Escr)}
        \arrow[hook, from=1-1, to=1-2]
        \arrow["{F\circ-}", from=1-2, to=1-3]
        \arrow["{i^{L}}", from=1-3, to=1-4]
    \end{tikzcd}$$
    with the identifications $\Cscr\otimes\Bscr=\Fun^{\kappa-\lex}((\Cscr^{\kappa})^{\Opp},\Bscr), \Cscr\otimes\Escr=\Fun^{\kappa-\lex}((\Cscr^{\kappa})^{\Opp},
    \Escr)$ and $i^{L}$ the left adjoint to the inclusion $i:\Fun^{\kappa-\lex}((\Cscr^{\kappa})^{\Opp},\Escr)\to\Fun((\Cscr^{\kappa})^{\Opp},\Escr)$
\end{definition}
\begin{remark}
    We can think of an oplax structure on a 2-functor as the data of a comparison map $\Cscr\otimes(F\circ G)\to(\Cscr\otimes F)\circ(\Cscr\otimes G)$. 
\end{remark}
\begin{definition}[Category of Correspondences]\label{def: category of correspondences}
    The category of correspondences $\Corr(\Bscr,\Escr)$ is the $\infty$-category of triples $(T,i_{1},i_{2})$ where $T$ is a presentable stable $\infty$-category $i_{1}:\Bscr\to T, i_{2}:\Escr\to T$ fully faithful and continuous, and $T$ admits a semiorthogonal decomposition $\langle i_{2}(\Escr), i_{1}(\Bscr)\rangle$. 
\end{definition}
We can define compositions of correspondences as follows. Suppose $T_{12}\in\Corr(\Bscr_{1},\Bscr_{2})$ and $T_{2,3}\in\Corr(\Bscr_{2},\Bscr_{3})$ we can define $T_{123}$ as the pushout $T_{12}\cup_{\Bscr_{2}}T_{23}$ in $\catPr^{\Lsf}_{\St}$ which has a semiorthogonal decomposition $\langle\Bscr_{3},\Bscr_{2},\Bscr_{1}\rangle$ and take $T_{13}$ as the subcategory generated by the images of $\Bscr_{1},\Bscr_{3}$ and as such $T_{13}\in\Corr(\Bscr_{1},\Bscr_{3})$. In fact we can show the following.
\begin{proposition}
    There is an equivalence of $\infty$-categories $\Fun^{\acc}(\Bscr,\Escr)\to\Corr(\Bscr,\Escr)$ by $(T, i_{1},i_{2})\mapsto (i_{2}^{R}i_{1}:\Dscr\to\Escr)$ and $(F:\Bscr\to\Escr)\mapsto \Escr\uplus_{F}\Bscr$ the semiorthogonal gluing of $\Escr,\Bscr$ along $F$. 
\end{proposition}
We will use the following proposition to prove the nontrivial direction of the equivalence of flatness and dualizability. 
\begin{proposition}
    Let $\Cscr\in\catPr^{\Lsf}_{\St}$. There is a natural oplax 2-functor $\Cscr\otimes-:\catPr^{\acc}_{\St}\to\catPr^{\acc}_{\St}$ inducing the Lurie tensor product on restriction to $\catPr^{\Lsf}_{\St}$. Furthermore, if $\Cscr$ is flat, then $\Cscr\otimes-:\catPr^{\acc}_{\St}\to\catPr^{\acc}_{\St}$ is a 2-functor. 
\end{proposition}
\begin{proof}[Proof Outline]
    We can define a oplax 2-endofunctor on correspondences on objects by the usual tensor product and on morphisms $(T,i_{1},i_{2})\mapsto (\Cscr\otimes T, \Cscr\otimes i_{1},\Cscr\otimes i_{2})$ and we have $\Cscr\otimes T_{123}=(\Cscr\otimes T_{12})\bigcup_{\Cscr\otimes\Bscr_{2}}(\Cscr\otimes T_{23})$ by commutativity of the tensor product with all colimits and thus with pushouts in $\catPr^{\Lsf}_{\St}$. This object is the target of a canonical map from $\Cscr\otimes T_{13}$ which is not fully faithful. As such, there is a map $\Cscr\otimes T_{13}\to(\Cscr\otimes T_{23})\circ(\Cscr\otimes T_{23})$. The precise argument uses the theory of complete 2-Segal spaces. If $\Cscr$ is flat, then the functor $\Cscr\otimes T_{13}\to\Cscr\otimes T_{123}$ is an actual 2-functor. Then using the equivalence between the 2-category of correspondences and the 2-category of accessible functors, we get the claim. 
\end{proof}

\begin{theorem}
    Let $\Cscr\in\catPr^{\Lsf}_{\St}$. The following are equivalent:
    \begin{enumerate}[label=(\alph*)]
        \item $\Cscr$ is flat. 
        \item $\Cscr$ is dualizable. 
    \end{enumerate}
\end{theorem}
\begin{proof}
    (b)$\Rightarrow$(a) If $\Cscr$ is dualizable then $\Cscr\otimes\Bscr=\Fun^{\Lsf}(\Cscr^{\vee},\Bscr)$ which preserves fully faithful functors. 

    (a)$\Rightarrow$(b) We will verify Grothendieck's AB6 axiom which is equivalent to dualizability per \Cref{thm: characterization of dualizable categories}. Consider a family of posets $J_{i}$ over an indexing set $i\in I$. Let $\Cscr\in\catPr^{\Lsf}_{\St}$ and consider a commutative square 
    $$% https://q.uiver.app/#q=WzAsNCxbMCwwLCJcXEZ1blxcbGVmdChcXHByb2Rfe2lcXGluIEl9Sl97aX0sXFxDc2NyXFxyaWdodCkiXSxbMiwwLCJcXENzY3IiXSxbMiwxLCJcXHByb2Rfe2lcXGluIEl9XFxDc2NyIl0sWzAsMSwiXFxwcm9kX3tpXFxpbiBJfVxcRnVuKEpfe2l9LFxcQ3NjcikiXSxbMCwxLCJcXFVjYWxee1xcY29saW19Il0sWzEsMiwiXFxkZWx0YSJdLFswLDMsIlxccHJvZF97aVxcaW4gSX1GIiwyXSxbMywyLCJcXHByb2Rfe2lcXGluIEl9XFxjb2xpbSIsMl1d
    \begin{tikzcd}
        {\Fun\left(\prod_{i\in I}J_{i},\Cscr\right)} && \Cscr \\
        {\prod_{i\in I}\Fun(J_{i},\Cscr)} && {\prod_{i\in I}\Cscr}
        \arrow["{\Ucal^{\colim}}", from=1-1, to=1-3]
        \arrow["{\prod_{i\in I}F}"', from=1-1, to=2-1]
        \arrow["\delta", from=1-3, to=2-3]
        \arrow["{\prod_{i\in I}\colim}"', from=2-1, to=2-3]
    \end{tikzcd}$$
    where $F$ is a left Kan extension. The AB6 axiom is equivalent to the fact that composition with the right adjoints of the vertical maps give isomorphisms of functors $\prod_{i\in I}\Fun(J_{i},\Cscr)\to\Cscr$. We can write this as the tensor product of $\Cscr$ with a diagram of spectra, and $\Sp$ being a dualizable category, AB6 holds so the Beck-Chevalley condition of the composition with right adjoints of the vertical functors. Then assuming flatness of $\Cscr$, we can apply $\Cscr\otimes-$ to show that it holds for $\Cscr$ showing $\Cscr$ is dualizable. 
\end{proof}
\begin{remark}
    The same works over a rigid $\EE_{1}$-(symmetric) monoidal base $\Escr$.
\end{remark}
We now want to consider some examples of non-dualizable categories. In particular, a natural question arises if dualizable categories are closed under extensions (of compactly generated categories). Consider, for example, a short exact sequence of categories 
$$% https://q.uiver.app/#q=WzAsNSxbMCwwLCIwIl0sWzEsMCwiXFxJbmQoXFxBc2NyKSJdLFsyLDAsIlxcQ3NjciJdLFszLDAsIlxcSW5kKFxcQnNjcikiXSxbNCwwLCIwIl0sWzAsMV0sWzEsMiwiRiJdLFsyLDMsIkciXSxbMyw0XV0=
\begin{tikzcd}
	0 & {\Ind(\Ascr)} & \Cscr & {\Ind(\Bscr)} & 0
	\arrow[from=1-1, to=1-2]
	\arrow["F", from=1-2, to=1-3]
	\arrow["G", from=1-3, to=1-4]
	\arrow[from=1-4, to=1-5]
\end{tikzcd}$$
with $F,G$ strongly continuous. Is it true that $\Cscr$ is dualizable? 

Note in the situation above $\Cscr$ has a semiorthogonal decomposition $\langle G^{R}(\Ind(\Bscr)), F(\Ind(\Ascr))\rangle$ so this corresponds to the composition $G^{\Rsf\Rsf}\circ F:\Ind(\Ascr)\to\Ind(\Bscr)$ where $G^{\Rsf\Rsf}$ is the twice right adjoint functor to $G$. Now observe that $\Fun^{\acc, \ex}(\Ind(\Ascr),\Ind(\Bscr))$ is equivalent to $\Fun^{\ex}(\Bscr,\Pro(\Ind(\Ascr)))$. We have the following proposition. 
\begin{proposition}\label{prop: dualizable iff tate objects}
    Let $$% https://q.uiver.app/#q=WzAsNSxbMCwwLCIwIl0sWzEsMCwiXFxJbmQoXFxBc2NyKSJdLFsyLDAsIlxcQ3NjciJdLFszLDAsIlxcSW5kKFxcQnNjcikiXSxbNCwwLCIwIl0sWzAsMV0sWzEsMiwiRiJdLFsyLDMsIkciXSxbMyw0XV0=
    \begin{tikzcd}
        0 & {\Ind(\Ascr)} & \Cscr & {\Ind(\Bscr)} & 0
        \arrow[from=1-1, to=1-2]
        \arrow["F", from=1-2, to=1-3]
        \arrow["G", from=1-3, to=1-4]
        \arrow[from=1-4, to=1-5]
    \end{tikzcd}$$
    be a short exact sequence of categories and $G^{\Rsf\Rsf}\circ F:\Ind(\Ascr)\to\Ind(\Bscr)$ the composition of the twice right adjoint of $G$ with $F$. The following are equivalent:
    \begin{enumerate}[label=(\alph*)]
        \item $\Cscr$ is dualizable. 
        \item $\Cscr$ is compactly generated. 
        \item The image of $\Bscr$ lies in the category of Tate objects of $\Pro(\Ind(\Ascr))$ -- the idempotent complete subcategory generated by $\Pro(\Ascr),\Ind(\Ascr)$ in $\Pro(\Ind(\Ascr))$. 
    \end{enumerate}
\end{proposition}
\begin{proof}
    See \cite[Rmk. A.3.13]{HermetianII}. 
\end{proof}
This description leads to a few examples of non-dualizable categories obtained by extensions as above. 
\begin{proposition}
    Let $k$ be a field and $\Cscr$ the category of triples $(V,W,\varphi)$ are dg $\CC$-vector spaces and $\varphi:\bigoplus_{\NN}V\to\bigoplus_{\NN}(W)$. Then $\Cscr$ is non-dualizable. 
\end{proposition}
\begin{proof}[Proof Outline]
    We need to show that $V\mapsto\prod_{\NN}\bigoplus_{\NN}V$ is not contained in the Tate objects of $\Perf(k)$ in the sense of \Cref{prop: dualizable iff tate objects} (c). Applying Grothendieck's AB6, we have 
    $$\prod_{\NN}\bigoplus_{\NN}V=\lim_{\{f:\NN\to\NN\}}\prod_{i\in \NN}\bigoplus_{j=0}^{f(i)-1}V=\lim_{\{f:\NN\to\NN\}}\Hom(X_{f},V)$$
    where $X_{f}=\prod_{i\in\NN}k^{f(i)}$. We want to show that $(\overline{X_{f}})_{f}$ is not an essentially constant system in $\Calk_{k}$ of $k$-vector spaces. For $f\leq g$ the map $X_{g}\to X_{f}$ is split surjective and admits a section. If $(\overline{X_{f}})_{f}$ is essentially constant, then it is eventually constant since the maps are split surjections. This fails since the fiber of $X_{f+1}\to X_{f}$ is a vector space of countable dimension which is nonzero in the Calkin category. 
\end{proof}

For $R$ a commutative Noetherian ring, a result of Neeman describes all the localizing subcategories of $\Dscr(R)$ as in bijection with subsets of $\spec(R)$. If $S\subseteq\spec(R)$ then $S$ corresponds to $\Dscr_{S}(R)$ which is a localizing subcategory generated by residue fields $\langle R_{\mathfrak{p}}/\mathfrak{p} R_{\mathfrak{p}}: \mathfrak{p}\in S\rangle$. 

This leads to the following. 
\begin{theorem}
    Let $R$ be a commutative Noetherian ring and $S\subseteq\spec(R)$. The following are equivalent:
    \begin{enumerate}[label=(\alph*)]
        \item $\Dscr_{S}(R)$ is dualizable. 
        \item $\Dscr_{S}(R)$ is compactly generated. 
        \item $S$ is convex -- for a short exact sequence $X\to Y\to Z$ with $X,Z\in S$ then $Y\in S$. 
    \end{enumerate}
\end{theorem}
\section{Lecture II (11th September)}\label{sec: Efimov II}
We turn to a discussion of localizing invariants of sheaves. In fact, the general case follows from a statement about sheaves on the real line. 
\begin{definition}[Sheaves with Non-Negative Singular Support]\label{def: sheaves with non-negative singular support}
     Let $\Fscr\in\Sh(\RR,\Sp)$. $\Fscr$ has non-negative singular support if for any $a<b$ $\Fscr((-\infty,b))\to\Fscr((a,b))$ is an isomorphism. 
\end{definition}
\begin{remark}
    Equivalently, the singular support of $\Fscr$ has singular support contained in the non-negative part of the cotangent bundle. 
\end{remark} 
We can then define the following category of non-negative sheaves. 
\begin{definition}[Non-Negative Sheaves]\label{def: non-negative sheaves}
    The category of non-negative sheaves on $\RR$ is the subcategory $\Sh_{\geq0}(\RR,\Sp)\subseteq\Sh(\RR,\Sp)$ such with non-negative singular support. 
\end{definition}
To be able to compute localizing invariants, we want to form a resolution by compactly generated categories which is given by the following proposition. 
\begin{proposition}\label{prop: SES for non-negative sheaves}
    There is a short exact sequence in $\Cat^{\dual}$:
    \begin{equation}\label{eqn: SES for non-negative sheaves}
        % https://q.uiver.app/#q=WzAsNSxbMSwwLCJcXFNoX3tcXGdlcTB9KFxcUlIsXFxTcCkiXSxbMCwwLCIwIl0sWzIsMCwiXFxGdW4oXFxRUV97XFxsZXF9XntcXE9wcH0sXFxTcCkiXSxbMywwLCJcXHByb2Rfe1xcUVF9XFxTcCJdLFs0LDAsIjAiXSxbMSwwXSxbMCwyXSxbMyw0XSxbMiwzXV0=
    \begin{tikzcd}
        0 & {\Sh_{\geq0}(\RR,\Sp)} & {\Fun(\QQ_{\leq}^{\Opp},\Sp)} & {\prod_{\QQ}\Sp} & 0
        \arrow[from=1-1, to=1-2]
        \arrow[from=1-2, to=1-3]
        \arrow[from=1-3, to=1-4]
        \arrow[from=1-4, to=1-5]
    \end{tikzcd}
    \end{equation}
\end{proposition}
\begin{proof}[Proof Outline]
    The essential content of the proof is that any sheaf $\Fscr\in\Sh_{\geq0}(\RR,\Sp)$ is determined by its value on rays $(-\infty,a)$ for $a\in\QQ$ and the sheaf condition reduces to $\Fscr((-\infty,a))=\lim_{b<a}\Fscr((-\infty,b))$ by \Cref{def: sheaves on a LCH space} (iii). Denoting the functor $\varphi:\Fun(\QQ_{\leq}^{\Opp},\Sp)\to\prod_{\QQ}\Sp$, we have $\varphi(\Fscr)_{a}=\mathrm{cone}(\colim_{b>a}\Fscr(b)\to\Fscr(a))$. Taking $\QQ$ to be a dense linearly ordered set in $\RR$, $\varphi$ is a localization of categories. $\varphi$ admits a right adjoint $\varphi^{R}$ defined by $\varphi^{R}((X_{a})_{a\in\QQ})(b)=X_{b}$ with transition maps given by zero maps. As such $\varphi\circ\varphi^{R}=\id_{\Fun(\QQ_{\leq}^{\Opp},\Sp)}$. We can now observe that $\Sh_{\geq0}(\RR,\Sp)=\varphi^{R}(\prod_{\QQ}\Sp)^{\perp}$ the right orthogonal of the essential image of the right adjoint, but this agrees with the left orthogonal since the inclusion of the category has both left and right adjoints giving an equivalence with the left orthogonal which is precisely the kernel of $\varphi$. 
\end{proof}
\begin{remark}
    $\varphi$ can be thought of as a variant of extension of scalars in the setting of almost mathematics. D. Vaintrob gives an interpretation of the middle term $\Fun(\QQ_{\leq}^{\Opp},\Sp)$ as quasicoherent sheaves on the infinite root stack and the functor $\varphi$ as a pullback to the infinite root stack. 
\end{remark}
Observe that the latter two terms of the exact sequence (\ref{eqn: SES for non-negative sheaves}) are compactly generated. Denoting 
\begin{equation}\label{def: compact objects A and B}
    \Ascr=\Fun(\QQ_{\leq}^{\Opp},\Sp)^{\omega}\text{ and }\Bscr=\left(\prod_{\QQ}\Sp\right)^{\omega}=\bigoplus_{\QQ}\Sp^{\omega},
\end{equation}
we can show the functor $\varphi^{\omega}:\Ascr\to\Bscr$ is a K-equivalence. 
\begin{definition}[K-Equivalence]\label{def: K-equivalence}
    Let $\varphi:\Ascr\to\Bscr$ be a functor. $f$ is a K-equivalence if there exists $\psi:\Bscr\to\Ascr$ such that $[\varphi\circ\psi]\cong[\id_{\Bscr}]\in K_{0}(\Fun(\Bscr,\Bscr))$ and $[\psi\circ\varphi]\cong[\id_{\Ascr}]\in K_{0}(\Fun(\Ascr,\Ascr))$. 
\end{definition} 
To show the desired K-equivalence statement, we will further require the following lemma concerning completed semiorthogonal decompositions. This lemma will also be pertinent to forthcoming discussions of the K-theory of nuclear modules. 
\begin{lemma}\label{lem: K0 commutes with limits along SOD}
    Let $\Cscr\in\Cat^{\Perf}$ be a small category with infinite semiorthogonal decomposition $\Cscr=\langle\Cscr_{0},\Cscr_{1},\dots\rangle$ and $\Bscr_{n}=\langle\Cscr_{0},\dots,\Cscr_{n}\rangle\subseteq\Cscr$. Then there are isomorphisms
    $$K_{0}(\lim_{n}\Bscr_{n})\cong\prod_{n}K_{0}(\Cscr_{n})\cong\lim_{n}K_{0}(\Bscr_{n}).$$
\end{lemma}
In other words, the formation of $K_{0}$ commutes with this especially nice limit. 
\begin{proof}
    There is a natural map $\lim_{n}\Bscr_{n}\to\prod_{\NN}\Cscr_{n}$. Denoting objects of $\Bscr_{n}$ by tuples $(x_{0},x_{1},\dots,x_{n})$ with $x_{i}\in\Cscr_{i}$ and $\Escr=\lim_{n}\Bscr_{n}$ there is a functor $F:\Escr\to\Cscr_{n}$ by $(x_{0},x_{1},\dots)\mapsto x_{n}$ and a functor $G:\prod_{\NN}\Cscr_{n}\to\Escr$ by $(x_{n})_{n\in\NN}\mapsto(0,\dots,0,x_{n},0,\dots,0)$ where all the transition maps are zero maps. $F\circ G\cong\id_{\Cscr_{n}}$ and we want to show $G\circ F\cong\id_{\Escr}$. We define endofunctors $\Phi_{n}:\Escr\to\Escr$ by $\Phi_{n}(x)=(0,\dots,x_{n},x_{n+1},\dots)$ with the transition maps as in $\Escr$. This yields an exact triangle in $\Fun(\Escr,\Escr)$ 
    $$% https://q.uiver.app/#q=WzAsMyxbMCwwLCJcXGJpZ29wbHVzX3tuXFxnZXExfVxcUGhpX3tufSJdLFsxLDAsIlxcYmlnb3BsdXNfe25cXGdlcTB9XFxQaGlfe259Il0sWzIsMCwiR1xcY2lyYyBGIl0sWzAsMV0sWzEsMl1d
    \begin{tikzcd}
        {\bigoplus_{n\geq1}\Phi_{n}} & {\bigoplus_{n\geq0}\Phi_{n}} & {G\circ F}
        \arrow[from=1-1, to=1-2]
        \arrow[from=1-2, to=1-3]
    \end{tikzcd}$$
    but then employing an Eilenberg Swindle-type argument, we can further observe that in K-groups 
    $$[G\circ F]\cong\left[\bigoplus_{n\geq0}\Phi_{n}\right]-\left[\bigoplus_{n\geq1}\Phi_{n}\right]\cong[\Phi_{0}]\cong[\id_{\Escr}].$$
\end{proof}
The result is as follows. 
\begin{proposition}\label{prop: K-equivalence of functors A to B}
    Let $\Ascr,\Bscr$ be as in (\ref{def: compact objects A and B}). The induced functor on compact objects $\varphi^{\omega}:\Ascr\to\Bscr$ K-equivalence.
\end{proposition}
\begin{proof}
    Let $\psi:\Bscr\to\Ascr$ given by $h_{a}$ for $a\in\QQ$ where $h_{a}$ are representable presheaves
    $$h_{a}(b)\begin{cases}
        \mathbb{S} & b\leq a \\
        0 & b>a
    \end{cases}$$
    and $\mathbb{S}$ is the sphere spectrum. This is a section of $\varphi^{\omega}$ so $\varphi\circ\psi\cong\id_{\Ascr}$. Conversely consider a bijection $\NN\to\QQ$ and let $\Ascr_{n}\subseteq\Ascr$ be sequences of representable presheaves $h_{a_{n}}$ as above so we have that $\Fun(\Ascr,\Ascr)\cong\lim_{n}\Fun(\Ascr_{n},\Ascr)$ but the restriction functors $\Fun(\Ascr_{n+1},\Ascr)\to\Fun(\Ascr_{n},\Ascr)$ have a fully faithful left and right adjoints. As such, from \Cref{lem: K0 commutes with limits along SOD} we have that $K_{0}(\Fun(\Ascr,\Ascr))\cong\lim_{n}K_{0}(\Fun(\Ascr_{n},\Ascr))=\mathrm{End}_{\ZZ}(\bigoplus_{\QQ}\ZZ)$ since $K_{0}(\mathbb{S})=\ZZ$. As such $[\psi\circ\varphi]=\id_{\Ascr}$.
\end{proof}
These results are sufficient to show that the image of non-negative sheaves on the real line under a localizing invariant is always trivial. 
\begin{theorem}\label{thm: trivial image of non-negative sheaves on the real line}
    For all localizing invariants $F:\Cat^{\Perf}\to\Escr$, $F^{\cont}(\Sh_{\geq0}(\RR,\Sp))=0$. 
\end{theorem}
\begin{remark}
    It is quite surprising that \Cref{thm: trivial image of non-negative sheaves on the real line} can be proved for arbitrary localizing invariants, without the additional hypothesis that the functor commutes with filtered colimits. 
\end{remark}
\begin{theorem}
    Let $F:\Cat^{\Perf}\to\Escr$ be a localizing invariant with $\Escr$ accessible and $X$ a finite CW complex. If $\Cscr\in\Cat^{\dual}_{\St}$ is a dualizable category, then $F^{\cont}(\Cscr)=F^{\cont}(\Sh(X,\Cscr))$. 
\end{theorem}

As a corollary, we have the following as a special case of the above. 
\begin{corollary}
    Let $R$ be an $\EE_{1}$-ring and $X$ a finite CW complex. Then $$K_{0}^{\cont}(\Sh(X,\Mod_{R}))=[X,\Omega^{\infty}K(R)].$$
\end{corollary}
Using the above, we can also show that $K$-theory commutes with infinite products. 

Turning to the category of sheaves on a locally compact Hausdorff space, we have the following.
\begin{theorem}\label{thm: criterion for isomorphism of localizing invariants}
    Let $F,G:\Cat^{\Perf}\to\Escr$ be localizing invariants with $\Escr$ having a non-degenerate $t$-structure and $\varphi: F\to G$ be a map in $\Fun(\Cat^{\Perf},
    \Escr)$. If $\varphi$ is an isomorphism on $\pi_{0}$, then $\varphi$ is an isomorphism in $\Fun(\Cat^{\Perf},\Escr)$.
\end{theorem}
\begin{proof}[Proof Outline]
    Noting that $\pi_{n}F(\Cscr)\cong\pi_{n+1}F(\Calk_{\omega_{1}}(\Cscr))$ and similarly for $G$, it follows by induction that $\pi_{n}\varphi$ is an isomorphism from $\pi_{n}F\to\pi_{n}G$ for $n\geq0$. The same proof applies in the continuous setting for $n\leq-1$. 

    For $\Cscr\in\Cat^{\dual}_{\St}$ there is a short exact sequence 
    $$% https://q.uiver.app/#q=WzAsNSxbMywwLCJcXENzY3IiXSxbNCwwLCIwIl0sWzIsMCwiXFxTaF97XFxnZXEwfShcXFJSLFxcU3ApIl0sWzEsMCwiXFxTaF97PjB9KFxcUlIsXFxTcCkiXSxbMCwwLCIwIl0sWzAsMSwiXFxHYW1tYV97Y30iXSxbNCwzXSxbMywyXSxbMiwwXV0=
    \begin{tikzcd}
        0 & {\Sh_{>0}(\RR,\Sp)} & {\Sh_{\geq0}(\RR,\Sp)} & \Cscr & 0
        \arrow[from=1-1, to=1-2]
        \arrow[from=1-2, to=1-3]
        \arrow[from=1-3, to=1-4]
        \arrow["{\Gamma_{c}}", from=1-4, to=1-5]
    \end{tikzcd}$$
    inducing an isomorphism $\pi_{n}F^{\cont}(\Cscr)\cong\pi_{n-1}(\Sh_{>0}(\RR,\Sp))$ by the vanishing of the middle term by \Cref{thm: trivial image of non-negative sheaves on the real line} so by induction $\pi_{n}\varphi$ is an isomorphism for all $n$. 
\end{proof}
\begin{corollary}
    The $K$-theory functor $K:\Cat^{\Perf}\to\Sp$ commutes with products. 
\end{corollary}
\begin{proof}
    Let $\Cscr\in\Cat^{\Perf}$ be a small category and a set $I$, the map $K(\prod_{I}\Cscr)\to\prod_{I}K(\Cscr)$ can be considered as a morphism of localizing invariants which induces an isomorphism on $K_{0}$ so the result follows from \Cref{thm: criterion for isomorphism of localizing invariants}.
\end{proof}
The following statement gives a variant of homotopy invariance. 
\begin{theorem}\label{thm: characterizations of localizing invariants}
    Let $X$ be a finite CW complex. If $F:\Cat^{\Perf}\to\Escr$ is a localizing invariant and $\Cscr\in\Cat^{\dual}_{\St}$ then $F^{\cont}(\Sh(X,\Cscr))\cong F^{\cont}(\Cscr)^{X}=\Gamma(X,F^{\cont}(\Cscr))$. 
\end{theorem}
\section{Lecture III (12th September)}\label{sec: Efimov III}
\section{Lecture IV (13th September)}\label{sec: Efimov IV}
\section{Lecture V (13th September)}\label{sec: Efimov V}
\part*{A. Matthew -- Dualizable Categories and their K-Theory}
\section{Lecture I (9th September)}\label{sec: Matthew I}

\section{Lecture II (9th September)}\label{sec: Matthew II}
We begin with some recollections on $\catPr^{\Lsf}$. See \cite[\S 5]{HTT} for further exposition. 

Recall the following definition. 
\begin{definition}[Presentable Category]\label{def: presentable category}
    An $\infty$-category $\Cscr$ is presentable if $\Cscr$ has all colimits and for some regular cardinal $\kappa$, the subcategory of $\kappa$-compact objects $\Cscr^{\kappa}$ is small and $\Cscr=\Ind_{\kappa}(\Cscr^{\kappa})$. 
\end{definition}
\begin{remark}
    Recall that the $\kappa$-compact objects are those $x\in\Cscr$ such that $\Hom_{\Cscr}(x,-):\Cscr\to\Grpd_{\infty}$ that preserve $\kappa$-filtered colimits and $\Cscr=\Ind_{\kappa}(\Cscr^{\kappa})$ states that $\Cscr$ is $\kappa$-compactly generated. 
\end{remark}
\begin{example}
    Let $\kappa=\omega_{0}$. If $\Cscr=\Ind(\Cscr^{\omega})$ then $\Cscr$ is a compactly generated category, the ind-completion of the subcategory of compact objects. For this to be presentable, the compact objects will have all finite colimits and thus $\Cscr$ will have all colimits. We will be most concerned with $\kappa=\omega_{1}$. 
\end{example}
We can now define $\catPr^{\Lsf}$ as follows. 
\begin{definition}[$\catPr^{\Lsf}$]\label{def: PrL}
    $\catPr^{\Lsf}$ is the $\infty$-category with objects presentable $\infty$-categories and morphisms colimit preserving functors. 
\end{definition}
\begin{remark}
    By the adjoint functor theorem, colimit preserving functors are left adjoints, justifying the $\Lsf$ in the notation. 
\end{remark}
We can in fact show that $\catPr^{\Lsf}$ has all limits and these limits are computed on the underlying $\infty$-categories. 
\begin{proposition}
    Let $\catPr^{\Lsf}$ be the $\infty$-category of presentable $\infty$-categories with left adjoint functors. Then the forgetful functor to (big) $\infty$-categories preserves limits and colimits in $\catPr^{\Lsf}$ are computed as limits in $\catPr^{\Lsf}$ on passage to right adjoint functors. 
\end{proposition}
More explicitly, a diagram in $\catPr^{\Lsf}$ is a diagram with transition maps left adjoint functors. Passage to right adjoints gives a diagram of presentable $\infty$-categories with transition maps right adjoint functors. Under the equivalence $\catPr^{\Lsf}\cong(\catPr^{\Rsf})^{\Opp}$ it then suffices to compute the limit on the underlying (big) $\infty$-categories. 
\begin{example}
    Let $\catPr^{\Lsf}_{\St}$ be the subcategory of $\catPr^{\Lsf}$ in \Cref{def: PrL} spanned by those stable categories. To compute the pushout 
    $$% https://q.uiver.app/#q=WzAsNCxbMCwwLCJcXEJzY3IiXSxbMiwwLCJcXENzY3IiXSxbMiwxLCJcXENzY3IvXFxCc2NyIl0sWzAsMSwiMCJdLFswLDNdLFszLDJdLFsxLDJdLFswLDFdXQ==
    \begin{tikzcd}
        \Bscr && \Cscr \\
        0 && {\Cscr/\Bscr}
        \arrow[from=1-1, to=1-3]
        \arrow[from=1-1, to=2-1]
        \arrow[from=1-3, to=2-3]
        \arrow[from=2-1, to=2-3]
    \end{tikzcd}$$
    in $\catPr^{\Lsf}_{\St}$ the limit of the passage to right adjoints then gives an isomorphism $\Cscr/\Bscr=\ker(\Cscr\to\Bscr)$ where the functor $\Cscr\to\Bscr$ is the right adjoint of the inclusion $\Bscr\hookrightarrow\Cscr$. 
\end{example}
Recall that $\catPr^{\Lsf}$ has a symmetric monoidal structure given by the Lurie tensor product where for $\Bscr,\Cscr\in\catPr^{\Lsf}$ the tensor product $\Bscr\otimes\Cscr$ is determined by the universal property that for any $\infty$-category $\Escr$, a map $\Bscr\times\Cscr\to\Escr$ preserving colimits in each factor of the source is uniquely extended from a colimit-preserving map $\Bscr\otimes\Cscr\to\Escr$. This is in fact a closed symmetric monoidal structure with the internal-hom $\Hom_{\catPr^{\Lsf}}(\Bscr,\Cscr)=\Fun^{\Lsf}(\Bscr,\Cscr)$ where $\Fun^{\Lsf}(\Bscr,\Cscr)$ are the colimit preserving functors between the categories. By unwinding the definitions, one finds $\Bscr\otimes\Cscr=\Fun^{\Lsf}(\Bscr,\Cscr^{\Opp})^{\Opp}$. 

One can observe that $\catPr^{\Lsf}$ has many idempotent algebra objects, those $\Ascr\in\catPr^{\Lsf}$ such that $\Ascr\otimes\Ascr\to\Ascr$ is an isomorphism. 
\begin{example}
    The $\infty$-category of spectra $\Sp$ is an idempotent object of $\catPr^{\Lsf}$. The modules over $\Sp$ are exactly the presentable stable $\infty$-categories $\catPr^{\Lsf}_{\St}\subseteq\catPr^{\Lsf}$.
\end{example}
As such, on restriction, $\catPr^{\Lsf}_{\St}$ is symmetric monoidal with unit $\Sp$. 

We can now define dualizable categories as follows. 
\begin{definition}
    Let $\Cscr\in\catPr^{\Lsf}_{\St}$. $\Cscr$ is dualizable if there exists $\Cscr^{\vee}=\Hom^{\Lsf}(\Cscr,\Sp)$ and maps $\ev:\Cscr\otimes\Cscr^{\vee}\to\Sp$ and $\coev:\Sp\to\Cscr\otimes\Cscr^{\vee}$ such that 
    $$% https://q.uiver.app/#q=WzAsNixbMCwwLCJcXENzY3IiXSxbMiwwLCJcXENzY3JcXG90aW1lc1xcQ3Njcl57XFx2ZWV9XFxvdGltZXNcXENzY3IiXSxbNCwwLCJcXENzY3IiXSxbMCwxLCJcXENzY3Jee1xcdmVlfSJdLFsyLDEsIlxcQ3Njcl57XFx2ZWV9XFxvdGltZXNcXENzY3JcXG90aW1lc1xcQ3Njcl57XFx2ZWV9Il0sWzQsMSwiXFxDc2NyXntcXHZlZX0iXSxbMCwxLCJcXGlkX3tcXENzY3J9XFxvdGltZXNcXGNvZXYiXSxbMSwyLCJcXGV2XFxvdGltZXNcXGlkX3tcXENzY3J9Il0sWzQsNSwiXFxpZF97XFxDc2NyXntcXHZlZX19XFxvdGltZXNcXGV2Il0sWzMsNCwiXFxjb2V2XFxvdGltZXNcXGlkX3tcXENzY3Jee1xcdmVlfX0iXV0=
    \begin{tikzcd}
        \Cscr && {\Cscr\otimes\Cscr^{\vee}\otimes\Cscr} && \Cscr \\
        {\Cscr^{\vee}} && {\Cscr^{\vee}\otimes\Cscr\otimes\Cscr^{\vee}} && {\Cscr^{\vee}}
        \arrow["{\id_{\Cscr}\otimes\coev}", from=1-1, to=1-3]
        \arrow["{\ev\otimes\id_{\Cscr}}", from=1-3, to=1-5]
        \arrow["{\coev\otimes\id_{\Cscr^{\vee}}}", from=2-1, to=2-3]
        \arrow["{\id_{\Cscr^{\vee}}\otimes\ev}", from=2-3, to=2-5]
    \end{tikzcd}$$
    are homotopic to the identities on $\Cscr,\Cscr^{\vee}$, respectively. 
\end{definition}
\begin{example}
    For $\Cscr=\Ind(\Cscr_{0})$ for $\Cscr_{0}\in\Cat^{\Perf}$, then $\Cscr$ is dualizable. In fact, $\Cscr^{\vee}=\Ind((\Cscr_{0})^{\Opp})$. For $\Cscr=\Dscr(A)$ the derived $\infty$-category of a discrete ring $A$, $\Cscr^{\vee}=\Dscr(A^{\Opp})$ where the evaluation map is the tensor product of modules and the coevaluation map is the diagonal bimodule. 
\end{example}
The dualizable categories amalgamate into an $\infty$-category $\Cat^{\dual}_{\St}$. 
\begin{definition}[$\Cat^{\dual}_{\St}$]\label{def: dualizable stable categories}
    $\Cat^{\dual}_{\St}$ is the $\infty$-category with objects dualizable categorise and morphisms strongly continuous functors -- colimit-preserving functors with colimit-preserving right adjoints. 
\end{definition}
\begin{remark}
    A continuous functor of compactly generated categories is strongly continuous if and only if it preserves compact objects. 
\end{remark}
\begin{example}
    There is a functor $\Ind:\Cat^{\Perf}\to\Cat^{\dual}_{\St}$ by $\Cscr\mapsto\Ind(\Cscr)$
\end{example}
In fact, we have the following characterization of dualizable categories. 
\begin{theorem}[Lurie, Clausen-Scholze, Efimov]\label{thm: characterization of dualizable categories}
    Let $\Cscr\in\catPr^{\Lsf}_{\St}$. The following are equivalent. 
    \begin{enumerate}
        \item $\Cscr$ is dualizable. 
        \item $\Cscr$ is a retract of a compactly generated category in $\catPr^{\Lsf}_{\St}$. 
        \item The colimit functor $\Ind(\Cscr)\to\Cscr$ sending an ind-object to its colimit admits a left adjoint. 
        \item $\Cscr$ is generated by compactly exhaustible objects. 
        \item There exists compactly generated $\Ascr,\Bscr\in\catPr^{\Lsf}_{\St}$ and a strongly continuous localization functor $L:\Ascr\to\Bscr$ -- admitting a colimit-preserving fully faithful right adjoint -- such that $\Cscr=\ker(L)$. 
        \item If $\Bscr\subseteq\Bscr'$ is a fully faithful inclusion in $\catPr^{\Lsf}_{\St}$ then $\Cscr\otimes\Bscr\to\Cscr\to\Bscr'$ is fully faithful.
        \item $\Cscr$ satisfies Grothendieck's (AB6) axiom: for an indexing set $I$ and a filtered category $J_{i}$ and functors $f_{i}:J_{i}\to\Cscr$ then 
        $$\prod_{i\in I}\colim_{j\in J_{i}}f_{i}(j_{i})=\colim_{(j_{i})\in\prod_{J_{i}},i\in I}\prod_{i\in I}f_{i}(j_{i}).$$
        \item There exists a fully faithful strongly continuous map $\Cscr\to\Bscr$ where $\Bscr\in\catPr^{\Lsf}_{\St}$ is compactly generated. 
    \end{enumerate}
\end{theorem}
\begin{proof}
    We will show 

    $(2)\Rightarrow(1)$ Retracts of dualizable objects are dualizable and compactly generated categories are dualizable. 

    $(2)\Rightarrow(3)$ (3) holds for compactly generated categories so \todo{App. D of Lurie's SAG.}

    $(3)\Rightarrow(8)$ The functor $\hat{\ysf}:\Cscr\to\Ind(\Cscr)$ factors through $\Ind(\Cscr^{\kappa})$. 

    $(8)\Rightarrow(2)$ If there is $i:\Cscr\to\Bscr$ is strongly continuous with $\Bscr$ compactly generated then $i^{R}i=\id_{\Cscr}$ and $\Cscr$ is a retract of $\Bscr$. 

    $(1)\Rightarrow(2)$ By presentability, there exists a Bousfield localization functor $\Ind(\Cscr^{\kappa})\to\Cscr$. We show this localization has a continuous section by showing that $\Fun^{\Lsf}(\Cscr,\Ind(\Cscr^{\kappa}))\to\Fun^{\Lsf}(\Cscr,\Cscr)$ is essentially surjective. By dualizability, $\Cscr^{\vee}\otimes\Ind(\Cscr^{\kappa})\to\Cscr^{\vee}\otimes\Cscr$ is essentially surjective, which holds since it is a Bousfield localization. 
\end{proof}
\section{Lecture III (10th September)}\label{sec: Matthew III}
\section{Lecture IV (11th September)}\label{sec: Matthew IV}
\section{Lecture V (12th September)}\label{sec: Matthew V}
%\part*{Research Talks}
\section{M. Ramzi (Copenhagen)}
\part*{End Matter}
\printbibliography
\end{document}
